%&pdflatex --translate-file=il2-pl
%Wzór dokumentu
%tu zmień marginesy i rozmiar czcionki
    \documentclass[a4paper,12pt]{article}
    \usepackage[margin=2.5cm]{geometry}

 %Lepiej tego nie zmieniaj, jak co to dodawaj pakiety
	\usepackage{titlesec}
	\usepackage{titling}
	\usepackage{fancyhdr}
	\usepackage{mdframed}
	\usepackage{graphicx}
	\usepackage{amsmath}
	\usepackage{amsfonts}
	\usepackage{rotating}
  \usepackage{listings}

%inny wygląd
	%\usepackage{tgbonum}


	%Zmienne, zmień je!
	\graphicspath{ {./ilustracje/} }
    \title{Baza Danych}
    \author{INF974658}
    \date{[data]}

  %lokalizacja polska (odkomentuj jak piszesz po polsku)

    \usepackage{polski}
    %\usepackage[polish]{babel}
    \usepackage{indentfirst}
	\usepackage{icomma}

    \brokenpenalty=1000
    \clubpenalty=1000
    \widowpenalty=1000

 %nie odkometowuj wszystkiego, użyj mózgu
    %\renewcommand\thechapter{\arabic{chapter}.}
	\renewcommand\thesection{\arabic{section}.}
	\renewcommand\thesubsection{\arabic{subsection}.}
	\renewcommand\thesubsubsection{\arabic{section}.\arabic{subsection}.\arabic{subsubsection}.}

%Makra
\newcommand{\obrazek}[2]{
	\begin{figure}[h]
		\centering
		\includegraphics[scale=#1]{#2}
	\end{figure}
}


    \newcommand{\twierdzonko}[1]{
        \begin{center}
        \begin{mdframed}
        #1
        \end{mdframed}
        \end{center}
    }

    \newcommand{\dwanajeden}[2]{
	\ensuremath \left( \begin{array}{c}
		#1\\
		#2
	\end{array} \right)
}

%Stopka i head (sekcja której nie powinno się zmieniać)
    \pagestyle{fancy}
    \fancyhead{}
    \fancyfoot{}

    %Zmieniaj od tego miejsca
    \lhead{\theauthor, \guillemotleft zlo\guillemotright}
	\rfoot{\thepage}
	\renewcommand{\headrulewidth}{1pt}
	\renewcommand{\footrulewidth}{1pt}
	
	\titleformat{\section}{\Large\bfseries}{\thesection}{0.5em}{}[\titlerule]
	\titleformat{\subsection}{\large}{\thesubsection}{0.5em}{}[\titlerule]

\begin{document}
\section*{Opis funkcji:}

Dana jest funkcja Pow\footnote{tu przepisana na język \emph{Python} z zachowaniem numeracji wierszy
z danego zapisu w pseudokodzie}:
\lstset{language = python,
	numbers = left,
	columns = fullflexible,
	frame = single}
	\begin{lstlisting}
def Pow(a, k):
	z = a
	y = 1
	m = k
	while m != 0:
		if m % 2 == 1:
			y = y*z
			
		m = int(m/2)
		z = z*z
		
	return y
	\end{lstlisting}
	
	Gdzie k jest wartością całkowitą nieujemną zapisaną na $n$ bitach, zatem:
	\[ k \in \left\langle 0 ; 2^n - 1 \right\rangle \wedge k \in Z\]
\subsection{Obliczyć pesymistyczną złożoność czasową przyjmując operację porównania w wierszu 6 funkcji
jako operację dominującą.}

Operacja porównania w wierszu 6 jest wywyoływana podczas każdej iteracji pętli w wierszu 5. Podczas każdej
iteracji tej pętli wartość $m$ jest dzielona przez 2 bez reszty w wierszu 9. Gdy wartość $m$ staje się równa 0
pętla przestaje się wykonywać. Na przykład:


\begin{minipage}[l]{0.3\textwidth}
\begin{align*}
		15 &\rightarrow 7 \\
		7 &\rightarrow 3 \\
		3 &\rightarrow 1 \\
		1 &\rightarrow 0 
\end{align*}
\end{minipage}
\begin{minipage}[l]{0.3\textwidth}
\begin{align*}
		9 &\rightarrow 4 \\
		4 &\rightarrow 2 \\
		2 &\rightarrow 1 \\
		1 &\rightarrow 0 
\end{align*}
\end{minipage}
\begin{minipage}[l]{0.3\textwidth}
\begin{align*}
		17 &\rightarrow 8 \\
		8 &\rightarrow 4 \\
		4 &\rightarrow 2 \\
		2 &\rightarrow 1 \\
		1 &\rightarrow 0 
\end{align*}
\end{minipage}


Liczby 9 i 15 da się zapisać na 4 bitach, ale liczbę 17 da się zapisać na minimalnie 5 bitach. Zatem pętla wykonuje
się dla dannego $k$ $n_m$ razy, gdzie $n_m$ to minimalna liczba bitów potrzebna do zapisania liczby $k$. 

Zatem pesymistyczna złożoność dla $n \in N^+$ to:
\[T_\text{pes}(n) = n\]

\subsection{Obliczyć średnią złożoność czasową przyjmując operację porównania w wierszu 6
funkcji jako operację dominującą.}

Z poprzedniego zadania wiemy że złożoność dla liczby zapisanej na minimalnej liczbie bitów jest
równa ilości jej bitów. Zatem złożoność średnia jest równa:

\begin{align*}
 T_\text{śr}(n) &= \frac{1 \cdot 1 + 2\cdot2 + 3 \cdot 4 + 4 \cdot 8 + \dots + n\cdot2^{n -  1}}{2 + 2 + 4 + 8 + \dots +
 2^{n - 1}} = \\
 &= \frac{1 \cdot 1 + 2\cdot2 + 3 \cdot 4 + 4 \cdot 8 + \dots + n\cdot2^{n -  1}}{2^n}
\end{align*}

\subsection{Obliczyć średnią liczbę wykonanych operacji mnożenia w wierszu 7 funkcji}

\subsection{Opisać  krótko,  jaki  warunek  musi  spełnić  wykładnik k aby  zachodził  przypadek pesymistyczny}

Z obliczeń z zadania pierwszego wiemy że złożoność dla liczby $k_4$ zapisanej na minimalnej liczbie bitów jest
równa ilości jej bitów. Zatem liczba bitów $k_4$ jest równa $n$.

Zatem:
\[ k_4 \in \left\langle 2^{n-1} - 1; 2^n - 1 \right\rangle \wedge k_4 \in Z\]

Aby wykładnik $k_4$ był przypadkiem pesymistycznym funkcji Pow.

\subsection{Obliczyć liczbę przypadków, dla których zachodzi przypadek pesymistyczny}

Z poprzedniego zadania wiemy, że przypadek pesymistyczny zachodzi dla liczby przypadków $k_5$,
która jest zależna od $n$
\begin{align*}
		k_5 &= 2^n - 1 - \left( 2^{n-1} - 1 \right) = 2^n - 2^{n- 1} = \\
		&= 2^{n-1}
\end{align*}

\end{document}
