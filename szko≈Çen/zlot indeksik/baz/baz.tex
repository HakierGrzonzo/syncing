%&pdflatex --translate-file=il2-pl
%Wzór dokumentu
%tu zmień marginesy i rozmiar czcionki
    \documentclass[a4paper,12pt]{article}
    \usepackage[margin=2.5cm]{geometry}

 %Lepiej tego nie zmieniaj, jak co to dodawaj pakiety
	\usepackage{titlesec}
	\usepackage{titling}
	\usepackage{fancyhdr}
	\usepackage{mdframed}
	\usepackage{graphicx}
	\usepackage{amsmath}
	\usepackage{amsfonts}
	\usepackage{rotating}
  \usepackage{listings}

%inny wygląd
	%\usepackage{tgbonum}


	%Zmienne, zmień je!
	\graphicspath{ {./ilustracje/} }
    \title{Baza Danych}
    \author{INF974658}
    \date{[data]}

  %lokalizacja polska (odkomentuj jak piszesz po polsku)

    \usepackage{polski}
    %\usepackage[polish]{babel}
    \usepackage{indentfirst}
	\usepackage{icomma}

    \brokenpenalty=1000
    \clubpenalty=1000
    \widowpenalty=1000

 %nie odkometowuj wszystkiego, użyj mózgu
    %\renewcommand\thechapter{\arabic{chapter}.}
	\renewcommand\thesection{\arabic{section}.}
	\renewcommand\thesubsection{\arabic{section}.\arabic{subsection}.}
	\renewcommand\thesubsubsection{\arabic{section}.\arabic{subsection}.\arabic{subsubsection}.}

%Makra
\newcommand{\obrazek}[2]{
	\begin{figure}[h]
		\centering
		\includegraphics[scale=#1]{#2}
	\end{figure}
}


    \newcommand{\twierdzonko}[1]{
        \begin{center}
        \begin{mdframed}
        #1
        \end{mdframed}
        \end{center}
    }

    \newcommand{\dwanajeden}[2]{
	\ensuremath \left( \begin{array}{c}
		#1\\
		#2
	\end{array} \right)
}

%Stopka i head (sekcja której nie powinno się zmieniać)
    \pagestyle{fancy}
    \fancyhead{}
    \fancyfoot{}

    %Zmieniaj od tego miejsca
    \lhead{\theauthor, \guillemotleft baz\guillemotright}
	\rfoot{\thepage}
	\renewcommand{\headrulewidth}{1pt}
	\renewcommand{\footrulewidth}{1pt}


\begin{document}

\section{Opis:}

Baza danych składa się z 4 tabel.
\begin{itemize}
		\item Modele Rolet - zawiera ofertę firmy ,,Roletka''. Zawiera:
				\begin{itemize}
						\item Numer modelu rolety.
						\item Nazwę modelu rolety.
						\item Opis modelu.
				\end{itemize}
		\item Rolety - zawiera informacje o wszystkich wyprodukowanych wariacji modeli rolet:
				\begin{itemize}
						\item Numer seryjny rolety.
						\item Numer modelu bazowego.
						\item Kolor materiału oraz kolor mocowań.
						\item numer użytego podmodelu wymiarowego.
				\end{itemize}
		\item Wymiary - zawiera wszystkie dostępne wymiary dla danego modelu rolety. W razie potrzeby
				można dodać więcej kolumn z innymi wymiarami. Wymiary są podawane jako wartości całkowite
				(w milimetrach). Jeśli wymiar posiada atrybut \emph{Nai\_zwmowienie} to jest on dostępny tylko
				na zamówienie i nie jest dostępny w produkcji seryjnej.
		\item Sprzedaż - zawiera informacje o sprzedanych roletach:
				\begin{itemize}
						\item Numer transakcji.
						\item Dane klienta (imię, nazwisko, adres).
						\item liczbę rolet.
						\item Numer seryjny rolet.
						\item cena.
				\end{itemize}
\end{itemize}

\section{Skrypt generujący:}

\lstset{language = SQL,
	numbers = left,
	columns = fullflexible,
	title = Skrypt \emph{mysql} generujący bazę danych:,
	frame = single}
\begin{lstlisting}
CREATE DATABASE rolety_baza;
USE rolety_baza;
CREATE TABLE modele_rolet(
    Numer_modelu int NOT NULL,
    nazwaModelu varchar(255),
    opis text,
    PRIMARY KEY (Numer_modelu)
);
CREATE TABLE rolety(
    id int NOT NULL,
    Numer_modelu int NOT NULL,
    kolor_materialu varchar(255),
    kolor_mocowan varchar(255),
    ID_wymiaru int NOT NULL,
    PRIMARY KEY (id)
);
CREATE TABLE wymiary(
    id int NOT NULL,
    id_modelu int NOT NULL,
    wymiar_rolety_mm_0 int,
    wymiar_rolety_mm_1 int,
    wymiar_rolety_mm_2 int,
    wymiar_rolety_mm_3 int,
    wymiar_rolety_mm_4 int,
    wymiar_rolety_mm_5 int,
    Na_zamowienie bool,
    PRIMARY KEY (id)
);
CREATE TABLE sprzedaz(
    numer_transakcji int NOT NULL,
    imie_klienta varchar(127),
    nazwisko_klienta varchar(127),
    adres_klienta varchar(511),
    liczba_rolet int,
    id_rolety int NOT NULL,
    cena DECIMAL(18, 2),
    PRIMARY KEY (numer_transakcji)
);
\end{lstlisting}



\end{document}
