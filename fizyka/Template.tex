%&pdflatex --translate-file=il2-pl
%Wzór dokumentu
%\usepackage{inputenc}[utf8]
%tu zmień marginesy i rozmiar czcionki
    \documentclass[a4paper,12pt]{article}
    \usepackage[margin=2.5cm]{geometry}
    
 %Lepiej tego nie zmieniaj, jak co to dodawaj pakiety
	\usepackage{titlesec}
	\usepackage{titling}
	\usepackage{fancyhdr}
	\usepackage{mdframed}
	\usepackage{graphicx}
	\usepackage{amsmath}
	\usepackage{amsfonts}
	\usepackage{tabularx}
	
%inny wygląd
	%\usepackage{tgbonum}
	
	
	%Zmienne, zmień je!
	\graphicspath{ {./images/} }
	\title{Badanie zależności okresu drgań wachadła matematycznego od jego długości.}
    \author{Grzegorz Koperwas}
	\date{\today{}}
    
  %lokalizacja polska (odkomentuj jak piszesz po polsku)
  
    \usepackage{polski}
    %\usepackage[polish]{babel} 
    \usepackage{indentfirst}
	\usepackage{icomma} 
	
    \brokenpenalty=1000
    \clubpenalty=1000
    \widowpenalty=1000    
 
 %nie odkometowuj wszystkiego, użyj mózgu
    %\renewcommand\thechapter{\arabic{chapter}.}
	\renewcommand\thesection{\arabic{section}.}
	\renewcommand\thesubsection{\arabic{section}.\arabic{subsection}.}
	\renewcommand\thesubsubsection{\arabic{section}.\arabic{subsection}.\arabic{subsubsection}.}

%Makra
    
	\newcommand{\obrazek}[3]{
	\begin{figure}[h]
		\centering
		\includegraphics[scale=#1]{#2}
		\caption{#3}
	\end{figure}
}     
            
    
    \newcommand{\twierdzonko}[1]{
        \begin{center}
        \begin{mdframed}
        #1
        \end{mdframed}          
        \end{center}
    } 
    
    \newcommand{\dwanajeden}[2]{
	\ensuremath \left( \begin{array}{c}
		#1\\
		#2
	\end{array} \right)
}  
      
%Stopka i head (sekcja której nie powinno się zmieniać)
    \pagestyle{fancy}
    \fancyhead{}
    \fancyfoot{}
    
    %Zmieniaj od tego miejsca
	\rfoot{\thepage}
	%\lfoot{Grzegorz Koperwas, \LaTeX note}
	\renewcommand{\headrulewidth}{0pt}
	\renewcommand{\footrulewidth}{1pt}
\titleformat{\section}{\Large \bfseries}{}{1em}{\thesection \hspace{0.5em}}[\titlerule]
    
\begin{document}
\begin{titlepage}
		\maketitle
\end{titlepage}


\section{Wstęp teoretyczny}

Celem doświadczenia było wyznaczenie przyspieszenia grawitacyjnego $g$ poprzez pomiar czasu w jakim wachadło wykona 10 cykli w zależności od długości wachadła. 
(tu ma być obrazek z tikz-a)

Z praw Newton'a:
\[F = ma\]

Gdzie F jest siłą wypadkową działającą na wachadło. Na wachadło działa siła ciężkości i naciągu sznurka. Siła naciągu jest równa składowej siły ciężkości prostopadłej do toru ruchu wachadła, zatem siła wypadkowa jest równa równoległej składowej siły ciężkości. Zatem:
\begin{align*}
		ma &= -mg\sin \alpha \\
		a &= -g\sin \alpha
\end{align*}

Przyspieszenie $a$ może zostać powiązane z zmianą kąta $\alpha$. Niech $s$ to długość łuku zakreślanego przez wachadło.
\begin{align*}
		s &= l\alpha\\
		v &= \frac{ds}{dt} = l \frac{d\alpha}{dt}\\
		a &= \frac{d^2s}{dt^2} = l \frac{d^2 \alpha}{d t^2}
\end{align*}
zatem:
\begin{align*}
		l \frac{d^2 \alpha}{d t^2} &= -g \sin \alpha
\end{align*}
\begin{equation}
		\frac{d^2 \alpha}{d t^2} + \frac{g}{l}\sin \alpha = 0
		\label{eq:preAprox}
\end{equation}

Dla małych kątów możemy założyć że $\sin \alpha\approx \alpha$, zatem po podstawieniu do równiania \ref{eq:preAprox} otrzymujemy równanie oscylatora harmonicznego:
\[ \frac{d^2 \alpha}{d t^2} + \frac{g}{l} \alpha = 0 \]

Dla warunków początkowych $\alpha \left( 0 \right) = \alpha_0$ i $\frac{d\alpha}{dt}\left( 0 \right) = 0$:
\[ \alpha \left( t \right) = \alpha_0 \cos\left(\sqrt{\frac{g}{l}}t \right) \]

Zatem okres jest równy:
\begin{equation}
		T = 2\pi \sqrt{\frac{l}{g}} \qquad \text{dla małych kątów}
		\label{eq:okres}
\end{equation}

Ostatecznie w celu uzyskania wykresu $T^2 \left(l\right)$:
\begin{equation}
		T^2\left( l \right) = \frac{4\pi^2}{g}  \cdot l
		\label{eq:wykres}
\end{equation}

\begin{table}[h]
		\centering
		\footnotesize{
		\begin{tabular}{|c|c|c|c|c|c|c|c|c|c|c|}
				\hline
	długość             &\multicolumn{10}{c|}{czas t (s) $\pm 0,01$s} \\ \cline{2-11}
	l (cm) $\pm 0,1$cm	& $t_1$ & $t_2$ & $t_3$ & $t_4$ & $t_5$ & $t_6$ & $t_7$ & $t_8$ & $t_9$ & $t_{10}$\\ \hline
		  80,0  & 18,32 & 17,72 & 18,28 & 18,01 & 18,29 & 18,18 & 17,87 & 17,72 & 18,03 & 18,18\\
		  70,0  & 16,66 & 16,73 & 16,84 & 16,93 & 16,67 & 16,67 & 16,83 & 16,91 & 16,68 & 16,67\\
		  60,0  & 15,37 & 15,43 & 15,71 & 15,52 & 15,55 & 15,28 & 15,48 & 15,54 & 15,50 & 15,59\\
		  50,0  & 14,16 & 14,14 & 14,17 & 14,25 & 14,13 & 14,29 & 14,32 & 14,16 & 14,23 & 14,23\\
		  40,0  & 12,77 & 12,55 & 12,61 & 12,54 & 12,86 & 12,68 & 12,64 & 12,72 & 12,61 & 12,63\\
		  30,0  & 11,03 & 10,98 & 10,95 & 10,94 & 10,95 & 10,73 & 10,93 & 11,06 & 10,87 & 10,87\\
		  20,0  & 8,87 & 8,97 & 9,02 & 9,01 & 8,97 & 8,99 & 8,91 & 8,88 & 8,93 & 8,84\\
		  10,0  & 6,09 & 6,27 & 6,18 & 6,20 & 6,25 & 6,49 & 6,18 & 6,33 & 6,25 & 6,27\\\hline
\end{tabular}
}
		\caption{Tabela wyników pomiarów}
		\label{tab:pomiar}
\end{table}

\section{Analiza wyników pomiarów}

Dla wyników doświadczenia w tabeli \ref{tab:pomiar} obliczamy:
\begin{itemize}
		\item Średni czas $\bar{t}$.
		\item Odchylenie standardowe średniego czasu $\Delta \bar{t}$.
		\item Okres wychyleń wachadła $T$.
		\item Niepewność okresu $\Delta T$.
		\item Kwadrat okresu $T^2$.
		\item Niepewność kwadratu okresu $\Delta \left( T^2 \right)$.

\end{itemize}
\subsection*{Odcylenie standardowe}
Odchylenie standardowe średniej jest równe:
\[ \Delta \bar{t} = \sqrt{\frac{\sum \limits^n_{i = 1}\left( x_i - \bar{x}\right)^2}{n\left(n-1 \right)}}\]
Przykładowo:
\begin{align*}
		&\sqrt{\frac{\left( 6,09- 6,25 \right)^2 +
		\left( 6,27- 6,25 \right)^2 +
		\left( 6,33- 6,25 \right)^2 +_{\dots} +
		\left( 6,25- 6,25 \right)^2 +
		\left( 6,27 - 6,25 \right)^2}{10 \cdot 9}} =\\
		&= \sqrt{\frac{0,03 + 0,00 + 0,01 + 0,00 + 0,00 + 0,06 + 0,01 + 0,01 + 0,00 + 0,00}{90}} = \\
		&\approx 0,03
\end{align*}

\subsection*{Okres}

Okres wychyleń wachadła obliczamy za pomocą wzoru:
\[ T = \frac{\bar{x}}{n}\]
gdzie $n$ to liczba wychyleń wykonanych przez wachadło w czasie mierzonym $t$.

Przykładowo:
\[ \frac{18,06}{10} \approx 1,80 \]
Niepewność okresu wychyleń obliczamy w sposób analogiczny.

\subsection*{Kwadrat okresu}

Niepewność $T^2$ jest obliczana w poniższy sposób:
\[\frac{\Delta \left(T^2\right)}{T^2} = \frac{\Delta T}{T} + \frac{\Delta T}{T} = 2 \cdot \frac{\Delta T}{T}\]

\begin{table}[h]
		\centering
		\begin{tabular}{|c|c|c|c|c|c|c|}
				\hline
				długość l (cm) $\pm 0,1$cm & $\bar{t}$ & $\Delta \bar{t}$ & $T$ & $\Delta T$ & $T^2$ & $\Delta \left( T^2 \right)$ \\ \hline 
  80,0  & 18,06 & 0,07 & 1,81 & 0,01 & 3,26 & 0,01\\
  70,0  & 16,76 & 0,03 & 1,68 & 0,00 & 2,81 & 0,01\\
  60,0  & 15,50 & 0,04 & 1,55 & 0,00 & 2,40 & 0,01\\
  50,0  & 14,21 & 0,02 & 1,42 & 0,00 & 2,02 & 0,00\\
  40,0  & 12,66 & 0,03 & 1,27 & 0,00 & 1,60 & 0,01\\
  30,0  & 10,93 & 0,03 & 1,09 & 0,00 & 1,19 & 0,01\\
  20,0  & 8,94 & 0,02 & 0,89 & 0,00 & 0,80 & 0,00\\
  10,0  & 6,25 & 0,03 & 0,63 & 0,00 & 0,39 & 0,01\\\hline 

		\end{tabular}
		\caption{Tabela obrobionych wyników}
		\label{tab:obrobione}
\end{table}

\subsection*{Wykres:}
\obrazek{0.55}{wykres}{Wykres $T^2 \left(l \right)$}

\section{Wnioski}
\subsection*{Obliczanie przyspieszenia grawitacyjnego:}

Nachylenie wykresu $m$ funkcji z równania \ref{eq:wykres} jest równe:
\[m = \frac{4\pi^2}{g} \]
zatem:
\begin{align*}
		g &= \frac{4\pi^2}{m}\\
		g &= \frac{4\pi^2}{0,0407 \cdot 10^2} \approx 9,70 \frac{m}{s^2}
\end{align*}
Z programu \emph{Logger Pro} niepewność $m$ wynosi $0,0002532 \, \frac{s^2}{\text{cm}}$ więc ostatecznie:
\[ g \approx \left( 9,70 \pm 0,03 \right) \, \frac{m}{s^2} \]

Dla Gliwic przyspieszenie grawitacyjne wynosi:
\[ g_{\text{std}} = 9,81 \, \frac{m}{s^2}\]
Zatem błąd względny otrzymanej wartości wynosi:
\begin{align*}
		\Delta g &= \frac{\left|g_{\text{std}} - g\right|}{g_{\text{std}}} \cdot 100 \text{\%} \\
		\Delta g &= \frac{| 9,81 - 9,70|}{9,81} \cdot 100 \text{\%} \approx\\
		&\approx 1,12 \text{\%}
\end{align*}

\subsection*{Możliwe źródła niepewności:}



\end{document}
