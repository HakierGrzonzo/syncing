%&pdflatex --translate-file=il2-pl
%Wzór dokumentu
\usepackage{inputenc}[utf8]
%tu zmień marginesy i rozmiar czcionki
    \documentclass[a4paper,12pt]{article}
    \usepackage[margin=3.5cm]{geometry}
    
 %Lepiej tego nie zmieniaj, jak co to dodawaj pakiety
	\usepackage{titlesec}
	\usepackage{titling}
	\usepackage{fancyhdr}
	\usepackage{mdframed}
	\usepackage{graphicx}
	\usepackage{amsmath}
	\usepackage{amsfonts}
	
%inny wygląd
	%\usepackage{tgbonum}
	
	
	%Zmienne, zmień je!
	\graphicspath{ {./ilustracje/} }
    \title{Maszyna Turinga}
    \author{INF974658}
    \date{[data]}
    
  %lokalizacja polska (odkomentuj jak piszesz po polsku)
  
    \usepackage{polski}
    %\usepackage[polish]{babel} 
    \usepackage{indentfirst}
	\usepackage{icomma} 
	
    \brokenpenalty=1000
    \clubpenalty=1000
    \widowpenalty=1000    
 
 %nie odkometowuj wszystkiego, użyj mózgu
    %\renewcommand\thechapter{\arabic{chapter}.}
	\renewcommand\thesection{\arabic{section}.}
	\renewcommand\thesubsection{\arabic{section}.\arabic{subsection}.}
	\renewcommand\thesubsubsection{\arabic{section}.\arabic{subsection}.\arabic{subsubsection}.}

%Makra
    
\newcommand{\obrazek}[2]{
	\begin{figure}[h]
		\centering
		\includegraphics[scale=#1]{#2}
	\end{figure}
}     
            
    
    \newcommand{\twierdzonko}[1]{
        \begin{center}
        \begin{mdframed}
        #1
        \end{mdframed}          
        \end{center}
    } 
    
    \newcommand{\dwanajeden}[2]{
	\ensuremath \left( \begin{array}{c}
		#1\\
		#2
	\end{array} \right)
}  
      
%Stopka i head (sekcja której nie powinno się zmieniać)
    \pagestyle{fancy}
    \fancyhead{}
    \fancyfoot{}
    
    %Zmieniaj od tego miejsca
    \lhead{\theauthor, \guillemotleft tur\guillemotright}
	\rfoot{\thepage}
	\renewcommand{\headrulewidth}{1pt}
	\renewcommand{\footrulewidth}{1pt}

    
\begin{document}

\section{Opis działania}

\subsection*{Opis stanów:}

\begin{itemize}
	\item $q_1$ - Stan początkowy, szkukam ,,a'' lub ,,b'', idąc w lewo. Jeżeli znajdę ,,a'' lub ,,b'' to idę w prawo i przechodzę w $q_2$.
	\item $q_2$ - Zmieniam komórkę taśmy na ,,X'' i przechodzę w $q_3$.
	\item $q_3$ - Szukam ,,b'' idąc w lewo. Jeżeli na taśmie jest ,,b'' to zmieniam je na ,,z'' i przechodzę w $q_4$.\\
		Jeżeli na taśmie jest ,,$\emptyset$'' to przechodzę w $q_5$.
	\item $q_4$ - Szukam cyfry lub ,,$\emptyset$''. Jeżeli znaleziona liczba nie jest dziewiątką to dodaje do niej 1.\\
		Jeżeli znalezłem ,,9'' to zmieniam je na ,,0'' i idę w lewo.\\
		Jeśli znalazłem ,,$\emptyset$'' to zmieniam go na ,,1'', idę w lewo i zmieniam stan na $q_3$. 
	\item $q_5$ - Szukam ,,X'', idąc w prawo. Gdy go znajdę to zmieniam stan na $q_6$.
	\item $q_6$ - Szukam ,,a'' idąc w lewo. Jeżeli na taśmie jest ,,a'' to przechodzę w $q_
\end{itemize}<++>

\end{document}
