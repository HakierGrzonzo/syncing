%&pdflatex --translate-file=il2-pl
%Wzór dokumentu
%tu zmień marginesy i rozmiar czcionki
	\documentclass[a4paper,12pt]{article}
	\usepackage[margin=2.5cm]{geometry}
    
	\usepackage{inputenc}[utf8]
 %Lepiej tego nie zmieniaj, jak co to dodawaj pakiety
	\usepackage{titlesec}
	\usepackage{tikz}
	\usepackage{titling}
	\usepackage{fancyhdr}
	\usepackage{mdframed}
	\usepackage{graphicx}
	\usepackage{amsmath}
	\usepackage{amssymb}
	\usepackage{amsfonts}
	
%inny wygląd
	%\usepackage{tgbonum}
	
	
	%Zmienne, zmień je!
	\graphicspath{ {./ilustracje/} }
    \title{Badanie przebiegu zmieności funkcji}
    \author{Grzegorz Koperwas}
    \date{\today}
    
  %lokalizacja polska (odkomentuj jak piszesz po polsku)
  
    \usepackage{polski}
    %\usepackage[polish]{babel} 
    \usepackage{indentfirst}
	\usepackage{icomma} 
	
    \brokenpenalty=1000
    \clubpenalty=1000
    \widowpenalty=1000    
 
 %nie odkometowuj wszystkiego, użyj mózgu
    %\renewcommand\thechapter{\arabic{chapter}.}
	\renewcommand\thesection{\arabic{section}.}
	\renewcommand\thesubsection{\arabic{section}.\arabic{subsection}.}
	\renewcommand\thesubsubsection{\arabic{section}.\arabic{subsection}.\arabic{subsubsection}.}

	\titleformat{\section}{\Large \bfseries}{}{1em}{\thesection \hspace{0.5em}}[\titlerule]

%Makra
	\newcommand{\fodx}{
		\ensuremath f \left( x \right)
	}
	\newcommand{\podx}{
		\ensuremath f' \left( x \right)
	}
	\newcommand{\dodx}{
		\ensuremath f'' \left( x \right)
	}
	\newcommand{\naw}[1]{
		\ensuremath \left( #1 \right)
	}

\newcommand{\obrazek}[2]{
	\begin{figure}[h]
		\centering
		\includegraphics[scale=#1]{#2}
	\end{figure}
}     
            
    
    \newcommand{\twierdzonko}[1]{
        \begin{center}
        \begin{mdframed}
        #1
        \end{mdframed}          
        \end{center}
    } 
    
    \newcommand{\dnj}[2]{
	\ensuremath  \begin{array}{c}
		#1\\
		#2
	\end{array} 
}  
      
%Stopka i head (sekcja której nie powinno się zmieniać)
    \pagestyle{fancy}
    \fancyhead{}
    \fancyfoot{}
    
    %Zmieniaj od tego miejsca
	\rfoot{\thepage}
	%\lfoot{Grzegorz Koperwas, \LaTeX note}
	\renewcommand{\headrulewidth}{0pt}
	\renewcommand{\footrulewidth}{1pt}

    
\begin{document}

\begin{titlepage}
	\begin{center}
		\Large{\textbf{I LICEUM OGÓLNOKSZTAŁCĄCE IM. EDWARDA DEMBOWSKIEGO W GLIWICACH}}

		\vspace{3cm}

		\theauthor

		\vspace{2.5cm}

		\textbf{Badanie przebiegu zmieności funkcji}

		\[f \left( x \right) = x^4 - 6x^2 + 8x + 24 \]

		\vfill

		Gliwce
		\vspace{0.5cm}

		\titlerule
		\vspace{0.6cm}

		2019

	\end{center}

\end{titlepage}

\section{Analiza wzoru funkcji}

Dana jest funkcja:
\[ f \left( x \right) = x^4  - 6x^2 + 8x + 24 \]

Funkcja jest wielomianem więc jej dziedzna jest zbiorem liczb rzeczywistych:
\[D_f = \mathbb{R} \]

Funkcja jest wielomianem więc jest ciągła w dziedzinie. Funkcja jest różniczkowalna w dziedzinie. Funkcja nie jest parzysta, nie jest nieparzysta.
\subsection{Miejsca zerowe:}
\[ p: \quad \pm 1, \pm 2, \pm 3, \pm 4, \pm 6, \pm 12, \pm 24\]

\begin{center}
	
\begin{tabular}{l|c|c|c|c|c}
                           & $1$ & $0$  & $-6$ & $8$  & $24$    \\ \hline

		$-2$ &     & $-2$ & $4$  & $4$  & $-24$   \\ \hline

                           & $1$ & $-2$ & $-2$ & $12$ & $0 = R$ \\  

\end{tabular}
\end{center}

\[\fodx =  \left(x + 2 \right) \cdot \left( x^3 - 2x^2 - 2x + 12 \right) \]

\begin{center}
	\begin{tabular}{l|c|c|c|c}
                           & $1$ & $-2$ & $-2$ & $12$    \\ \hline

		$-2$	&  & $-2$& $8$  & $-12$   \\ \hline

                           & $1$ & $-4$ & $6$  & $0 = R$ \\ 

\end{tabular}
\end{center}
\begin{minipage}[c]{0.7\textwidth}
	\[\fodx =  \left(x + 2 \right)^2 \cdot \underbrace{\left( x^2 - 4x + 6 \right)}_{\Delta < 0} \]
\end{minipage}
\hspace{0.15cm} \vline \hspace{0.15cm}
\begin{minipage}[c]{0.2\textwidth}
	\begin{align*}
		\Delta &= 16 - 4 \cdot 6 = \\
		&= -8 < 0
	\end{align*}
\end{minipage}
\vspace{0.5cm}

Funkcja posiada miejsce zerowe w punkcie $\left(-2; 0 \right)$ drugiego stopnia. 

\subsection{Przecięcie z osią $OY$}

\[f \left( 0 \right) = 0^4  - 6\cdot 0^2 + 8\cdot 0 + 24 = 24 \]

Funkcja przecina oś $0Y$ w punkcie $ \left( 0; \, 24 \right)$.

\subsection{Granice na krańcach dziedziny}

\begin{align*}
	\lim\limits_{x \to - \infty} f \left( x \right) &=\lim\limits_{x \to - \infty} \underbrace{x^4}_{\to + \infty} \cdot \left( 1 - \underbrace{\frac{6}{x^2} + \frac{8}{x^3} + \frac{24}{x^4}}_{\to 0} \right) =\\
	&= + \infty \\
	\lim\limits_{x \to + \infty} f\left( x \right) &= \lim\limits_{x \to +\infty} \underbrace{x^4}_{\to + \infty} \cdot \left( 1 - \underbrace{\frac{6}{x^2} + \frac{8}{x^3} + \frac{24}{x^4}}_{\to 0} \right) = \\
	&= + \infty
\end{align*}

Funkcja nie posiada asymptot. 

\section{Analiza pierwszej pochodnej}
Obliczamy pierwszą pochodną:
\[ f' \left( x \right) = 4x^3 - 12x + 8 \]

Pierwsza pochodna danej funkcji jest wielomianem więc jej dzidziną jest zbiór liczb rzeczywistych.
\[ D_{f'} = \mathbb{R} \]

\subsection{Miejsca zerowe pochodnej}
\[ p: \pm 1, \pm 2, \pm 4, \pm 8\]
\begin{center}
	\begin{tabular}{l|c|c|c|c}		
		& $4$	& $0$	& $-12$	& $8$	\\ \hline
	$1$	&	& $4$	& $4$	& $-8$	\\ \hline
		& $4$	& $4$	& $-8$	& $0 = R$\\ 
	\end{tabular}
\end{center}
\begin{align*}
	\podx &= \left( x - 1 \right) \cdot \left( 4x^2 + 4x - 8 \right) \\
	4x^2 &+ 4x - 8 = 0 \\
	x^2 &+ x - 2 = 0 \\
	\Delta &= 1 + 4 \cdot 2 \cdot 1 = 9 > 0\\
	\sqrt{\Delta} &= \sqrt{9} = 3\\
	x_{1, 2} &= \frac{-1 \pm 3}{2} \Rightarrow x_1 = -2, \, x_2 = 1\\
	\podx &= \left( x - 1 \right)^2 \cdot \left( x + 2 \right)
\end{align*}

Pochodna funkcji posiada miejsce zerowe w $1$ drugiego stopnia. Pochodna funkcji posiada miejsce zerowe w punkcie $-2$. 

\subsection{Wyznaczanie ekstremów funkcji}
\subsubsection*{Warunek konieczny ekstremum}

\begin{align*}
	\podx &= 0 \Leftrightarrow \left( x - 1 \right)^2 \cdot \left( x + 2 \right)  = 0\\
	\podx &= 0 \Leftrightarrow x \in \left\{ -2; 1 \right\}
\end{align*}
Podejrzewam że funkcja ma ekstremum w $x_0 = -2$ oraz w $x_1 = 1$.
\begin{figure}[h]
	\centering
	\begin{tikzpicture}
		\draw (0,3)[thick,->] -- (10,3) node[anchor=north west] {$x$};
		\draw (2,0) -- (3,3);
		\draw (3,3) .. controls (3.8,5) and (5.2,5) .. (6,3);
		\draw (6,3) -- (8,8) node[anchor=north west] {$\podx$};
		\draw[thick] (3, 3.15) -- (3, 2.85) node[anchor = north] {$-2$};
		\draw[thick] (6, 3.15) -- (6, 2.85) node[anchor = north] {$1$} ;
		\node[anchor = center] at (4.5,3.85){$+$};
		\node[anchor = center] at (2,2.15){$-$};
		\node[anchor = center] at (7,3.85){$+$};
	\end{tikzpicture}
\end{figure}

\[ \left\{
	\begin{array}{l}
		\podx > 0 \Leftrightarrow x \in \left( -2; 1 \right) \cup \left( 1; + \infty \right) \\
		\podx < 0 \Leftrightarrow x \in \left( - \infty; -2 \right)
	\end{array}
\right. \]

\subsubsection*{Warunek wystarczający ekstremum}

\[ \left.
		\begin{array}{l}
			x \in S^- \left( -2, \delta \right) \Rightarrow \podx < 0 \\
			x \in S^+ \left( -2, \delta \right) \Rightarrow \podx > 0
		\end{array}
	\right\} \Rightarrow f \left( -2 \right) = 0 = f_{\text{min}}
\]
Funkcja posiada minimum w punkcie $\left( -2; 0 \right)$.

\[ \left.
		\begin{array}{l}
			x \in S^- \left( 1, \delta \right) \Rightarrow \podx > 0 \\
			x \in S^+ \left( 1, \delta \right) \Rightarrow \podx > 0
		\end{array}
	\right\} \Rightarrow 
\]
$\Rightarrow$ funkcja nie posiada ekstremum w $\left( 1; 27 \right)$.

\subsection{Monotoniczność funkcji}

\[ \left\{
	\begin{array}{l}
		\podx > 0 \Leftrightarrow x \in \left( -2; 1 \right) \cup \left( 1; + \infty \right) \\
		\podx < 0 \Leftrightarrow x \in \left( - \infty; -2 \right)
	\end{array}
\right. \]
więc:
\begin{itemize}
	\item Funkcja jest rosnąca w przedziale $\left\langle -2; +\infty \right)$.
\item Funkcja jest malejąca w przedziale $\left( -\infty ; -2 \right\rangle$.
\end{itemize}

\section{Analiza drugiej pochodnej}

\[ \dodx = 12x^2 - 12\]

Druga pochodna funkcji jest wielomianem, więc dziedzina drugiej pochodnej to zbiór liczb rzeczywistych.
\[ D_{\dodx} = \mathbb{R} \]

\subsection{Punkty przegięcia}

\subsubsection*{Warunek konieczny punktu przegięcia}

\begin{align*}
	\dodx = 0 &\Leftrightarrow 12x^2 - 12 = 0\\
	x^2 -1 &= 0 \\
	\left( x - 1 \right) \cdot \left( x + 1 \right) &= 0\\
	x = 1 &\vee x = - 1 \Rightarrow \\
	\Rightarrow x &\in \left\{ -1; 1 \right\}
\end{align*}

\begin{figure}[h]
	\centering
	\begin{tikzpicture}
		\draw[thick, ->] (0,3) -- (8,3);
		\draw (1,5) parabola bend (4,1) (7, 5);
	\end{tikzpicture}
\end{figure}

\[\left\{
	\begin{array}{l}
		\dodx > 0 \Leftrightarrow x \in \left( -\infty; -1 \right) \cup \left( 1; +\infty \right)\\
		\dodx < 0 \Leftrightarrow x \in \left( -1; 1 \right)
	\end{array}
\right.
\]

Więc:
\begin{itemize}
	\item Krzywa jest wypukła w $\left( -\infty; -1 \right) \cup \left( 1; +\infty \right)$
	\item Krzywa jest wklęsła w $\left( -1; 1 \right)$

\end{itemize}

\subsubsection*{Warunek wystarczający punktu przegięcia}

\[ \left.
		\begin{array}{l}
			x \in S^- \left(-1; \delta \right) \Rightarrow \dodx > 0 \\
			x \in S^+ \left(-1; \delta \right) \Rightarrow \dodx < 0
		\end{array}
	\right\} \Rightarrow P_1 = \left(-1; f \left(-1 \right) \right) = \left( -1; 11 \right) \text{ p.p.}
\]
Funkcja posiada punkt przegięcia w $\left( -1; 11 \right)$.

\[ \left.
		\begin{array}{l}
			x \in S^- \left(1; \delta \right) \Rightarrow \dodx < 0 \\
			x \in S^+ \left(1; \delta \right) \Rightarrow \dodx > 0
		\end{array}
	\right\} \Rightarrow P_2 = \left(1; f \left(1 \right) \right) = \left( 1; 27 \right) \text{ p.p.}
\]
Funkcja posiada punkt przegięcia w $\left( 1; 27 \right)$.

\section{Tabela przebiegu zmienności funkcji}

\begin{table}[h]
	\centering
	\begin{tabular}{c|c|c|c|c|c|c|c|c|c|}
		$x$	& $\naw{ -\infty; -2}$	& $-2$	& $\naw{ -2; -1}$	& $-1$	& $\naw{-1; 0}$	& $0$	& $\naw{0;1}$	& $1$	& $\naw{1; +\infty}$	\\ \hline 
		$\podx$	& $-$			& $0$	& $+$			& $+$	& $+$		& $+$	& $+$		& $0$	& $+$			\\ \hline
		$\dodx$	& $+$			& $+$	& $+$			& $0$	& $-$		& $-$	& $-$		& $0$	& $+$			\\ \hline
		$\fodx$	& $^{+\infty} \searrow_{\:0}$	& $\dnj{\min}{0}$	& $_0 \nearrow^{\: 11}$	& $11$	& $_{11}\nearrow^{\: 24}$	& $24$	& $_{24}\nearrow^{\: 27}$	& $27$	& $_{27}\nearrow^{\: + \infty}$ \\
			& $\smile$		& $\smile$	& $\smile$	& 	& $\frown$	& $\frown$	& $\frown$ 	&	& $\smile$
	\end{tabular}
	\caption{Tabela zmienności funkcji $\fodx = x^4 - 6x^2 + 8x + 24$}
	\label{tab:tabelken}
\end{table}


\end{document}
