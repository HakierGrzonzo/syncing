%&pdflatex --translate-file=il2-pl
%Wzór dokumentu
%tu zmień marginesy i rozmiar czcionki
    \documentclass[a4paper,12pt]{article}
    \usepackage[margin=2.5cm]{geometry}
    
 %Lepiej tego nie zmieniaj, jak co to dodawaj pakiety
	\usepackage{titlesec}
	\usepackage{titling}
	\usepackage{fancyhdr}
	\usepackage{mdframed}
	\usepackage{graphicx}
	\usepackage{amsmath}
	\usepackage{amsfonts}
	\usepackage{rotating}
	
%inny wygląd
	%\usepackage{tgbonum}
	
	
	%Zmienne, zmień je!
	\graphicspath{ {./ilustracje/} }
    \title{Maszyna Turinga}
    \author{INF974658}
    \date{[data]}
    
  %lokalizacja polska (odkomentuj jak piszesz po polsku)
  
    \usepackage{polski}
    %\usepackage[polish]{babel} 
    \usepackage{indentfirst}
	\usepackage{icomma} 
	
    \brokenpenalty=1000
    \clubpenalty=1000
    \widowpenalty=1000    
 
 %nie odkometowuj wszystkiego, użyj mózgu
    %\renewcommand\thechapter{\arabic{chapter}.}
	\renewcommand\thesection{\arabic{section}.}
	\renewcommand\thesubsection{\arabic{section}.\arabic{subsection}.}
	\renewcommand\thesubsubsection{\arabic{section}.\arabic{subsection}.\arabic{subsubsection}.}

%Makra
	\newcommand{\tur}[2]{
			$q_{#1},$ #2
	}
    
	\renewcommand{\nu}{
			$\emptyset$
	}
\newcommand{\obrazek}[2]{
	\begin{figure}[h]
		\centering
		\includegraphics[scale=#1]{#2}
	\end{figure}
}     
            
    
    \newcommand{\twierdzonko}[1]{
        \begin{center}
        \begin{mdframed}
        #1
        \end{mdframed}          
        \end{center}
    } 
    
    \newcommand{\dwanajeden}[2]{
	\ensuremath \left( \begin{array}{c}
		#1\\
		#2
	\end{array} \right)
}  
      
%Stopka i head (sekcja której nie powinno się zmieniać)
    \pagestyle{fancy}
    \fancyhead{}
    \fancyfoot{}
    
    %Zmieniaj od tego miejsca
    \lhead{\theauthor, \guillemotleft tur\guillemotright}
	\rfoot{\thepage}
	\renewcommand{\headrulewidth}{1pt}
	\renewcommand{\footrulewidth}{1pt}

    
\begin{document}

\section{Opis działania}

\subsection*{Opis stanów:}

\begin{itemize}
	\item $q_1$ - Stan początkowy, szkukam ,,a'' lub ,,b'', idąc w lewo. Jeżeli znajdę ,,a'' lub ,,b'' to idę w prawo i przechodzę w $q_2$.
	\item $q_2$ - Zmieniam komórkę taśmy na ,,X'' i przechodzę w $q_3$.
	\item $q_3$ - Szukam ,,b'' idąc w lewo. Jeżeli na taśmie jest ,,b'' to zmieniam je na ,,z'' i przechodzę w $q_4$.\\
		Jeżeli na taśmie jest ,,$\emptyset$'' to przechodzę w $q_5$.
	\item $q_4$ - Szukam cyfry lub ,,$\emptyset$''. Jeżeli znaleziona liczba nie jest dziewiątką to dodaje do niej 1.\\
		Jeżeli znalezłem ,,9'' to zmieniam je na ,,0'' i idę w lewo.\\
		Jeśli znalazłem ,,$\emptyset$'' to zmieniam go na ,,1'', idę w lewo i zmieniam stan na $q_3$. 
	\item $q_5$ - Szukam ,,X'', idąc w prawo. Gdy go znajdę to zmieniam stan na $q_6$.
	\item $q_6$ - Szukam ,,a'' idąc w lewo.
			Jeżeli na taśmie jest ,,a'' to zamieniam je na ,,z'' i  przechodzę w $q_7$.\\
			Jeżeli znajduje ,,$\emptyset$'' to przechodzę w $q_8$.
	\item $q_7$ - Szukam cyfry, idąc w prawo. Jeśli znajdę cyfrę to odejmuję od niej jeden. \\
			Jeżeli liczba jest równa 0 to zamieniam ją na 9 i idę w prawo.\\
			Jeżeli znajdę ,,$\emptyset$'' to przechodzę w $q_9$.
	\item $q_8$ - Szukam cyfry różnej od zera, idąc w prawo. Jeżeli ją znajdę to przechodzę w $q_{10}$.\\
			Jeżeli znajdę ,,$\emptyset$'' lub 0 to przechodzę w $q_{11}$.
\end{itemize}
\subsubsection*{Stany końcowe:}
\begin{itemize}
	\item $q_9$ - szukam ,,X'' idąc w prawo, zmieniam go na A.
	\item $q_{10}$ - szukam ,,X'' idąc w lewo, zmieniam go na B.
	\item $q_{11}$ - szukam ,,X'' idąc w lewo, zmieniam go na N.
\end{itemize}

\section{Tabela}
\begin{sidewaystable}[h]
		\centering
		\begin{tabular}{c|c|c|c|c|c|c|c|c|c|c|c|}
				&	$q_1$	&	$q_2$	&	$q_3$	&	$q_4$	&	$q_5$	&	$q_6$	&	$q_7$	&	$q_8$	&	$q_9$	&	$q_{10}$	&	$q_{11}$	\\ \hline

			a	&	\tur{2}{P, a}	&	&	\tur{3}{L, a}	&	\tur{4}{P, a}	&	\tur{5}{P, a}	&	\tur{7}{P, z}	&	\tur{7}{P, a}	&	\tur{8}{P, a}	&&&\\\hline

			b	&	\tur{2}{P, b}	&	&	\tur{4}{P, z}	&	\tur{4}{L, b}	&	\tur{5}{P, b}	&	\tur{6}{L, b}	&	\tur{7}{P, b}	&	\tur{8}{P, b}	&&&\\\hline

			X	&	&	&	\tur{3}{L, X}	&	\tur{4}{P, X}	&	\tur{6}{L, X}	&	\tur{6}{L, X}	&	\tur{7}{P, X}	&	\tur{8}{P, X}	&	\tur{9}{0, A}
			&	\tur{10}{0, B}	& \tur{11}{0, N} \\\hline

			z	&	&	&	\tur{3}{L, z}	&	\tur{4}{P, z}	&	\tur{5}{P, z}	&	\tur{6}{L, z}	&	\tur{7}{P, z}	&	\tur{8}{P, z}	& \tur{9}{P, z}	&&\\\hline
			
			$\emptyset$	&	\tur{1}{L, \nu}	&	\tur{3}{L, X}	&	\tur{5}{P, \nu}	&	\tur{3}{L, 1}	&	\tur{5}{P, \nu}	&	\tur{8}{P, \nu}	&	\tur{9}{L, \nu}	&
			\tur{11}{L, \nu}	&	\tur{9}{P, \nu}	&	\tur{10}{L, \nu}	&	\tur{11}{L, \nu} \\\hline
			
			0   &   &   &  \tur{3}{L, 0}    &   \tur{3}{L, 1}   &   \tur{5}{P, 0}   &   \tur{6}{L, 0}   &   \tur{7}{P, 9}   &   \tur{11}{L, 0}  &   &   \tur{10}{L, 0}  &   \tur{11}{L, 0} \\\hline

1   &   &   &  \tur{3}{L, 1}    &   \tur{3}{L, 2}   &   \tur{5}{P, 1}   &   \tur{6}{L, 1}   &   \tur{6}{L, 0}   &   \tur{10}{L, 1}  &   &   \tur{10}{L, 1}  &   \tur{11}{L, 1} \\\hline

2   &   &   &  \tur{3}{L, 2}    &   \tur{3}{L, 3}   &   \tur{5}{P, 2}   &   \tur{6}{L, 2}   &   \tur{6}{L, 1}   &   \tur{10}{L, 2}  &   &   \tur{10}{L, 2}  &   \tur{11}{L, 2} \\\hline

3   &   &   &  \tur{3}{L, 3}    &   \tur{3}{L, 4}   &   \tur{5}{P, 3}   &   \tur{6}{L, 3}   &   \tur{6}{L, 2}   &   \tur{10}{L, 3}  &   &   \tur{10}{L, 3}  &   \tur{11}{L, 3} \\\hline

4   &   &   &  \tur{3}{L, 4}    &   \tur{3}{L, 5}   &   \tur{5}{P, 4}   &   \tur{6}{L, 4}   &   \tur{6}{L, 3}   &   \tur{10}{L, 4}  &   &   \tur{10}{L, 4}  &   \tur{11}{L, 4} \\\hline

5   &   &   &  \tur{3}{L, 5}    &   \tur{3}{L, 6}   &   \tur{5}{P, 5}   &   \tur{6}{L, 5}   &   \tur{6}{L, 4}   &   \tur{10}{L, 5}  &   &   \tur{10}{L, 5}  &   \tur{11}{L, 5} \\\hline

6   &   &   &  \tur{3}{L, 6}    &   \tur{3}{L, 7}   &   \tur{5}{P, 6}   &   \tur{6}{L, 6}   &   \tur{6}{L, 5}   &   \tur{10}{L, 6}  &   &   \tur{10}{L, 6}  &   \tur{11}{L, 6} \\\hline

7   &   &   &  \tur{3}{L, 7}    &   \tur{3}{L, 8}   &   \tur{5}{P, 7}   &   \tur{6}{L, 7}   &   \tur{6}{L, 6}   &   \tur{10}{L, 7}  &   &   \tur{10}{L, 7}  &   \tur{11}{L, 7} \\\hline

8   &   &   &  \tur{3}{L, 8}    &   \tur{3}{L, 9}   &   \tur{5}{P, 8}   &   \tur{6}{L, 8}   &   \tur{6}{L, 7}   &   \tur{10}{L, 8}  &   &   \tur{10}{L, 8}  &   \tur{11}{L, 8} \\\hline

9   &   &   &  \tur{3}{L, 9}    &   \tur{4}{P, 0}   &   \tur{5}{P, 9}   &   \tur{6}{L, 9}   &   \tur{6}{L, 8}   &   \tur{10}{L, 9}  &   &   \tur{10}{L, 9}  &   \tur{11}{L, 9} \\\hline
		\end{tabular}
		\caption{Tabela charakterystyczna maszyny Turinga}
\end{sidewaystable}


\end{document}
